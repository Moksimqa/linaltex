\documentclass[../main.tex]{subfiles}
\begin{document}
\lecture{13}{28.04}{}

\section{Определение и примеры унитарных и евклидовых пространств}
\begin{definition}
    $\V$ - ЛП над $\R$ называется евклидовым (вещественным евклидовым пространством - ВЕП), если в нем определена функция $(,): \V \times \V\to \R: \forall x,y ,z \in \V$ и $\forall \alpha,\beta \in \R$ выполнены следующие свойства:\\ 
    $\begin{aligned}
        &1^{\circ} (x,x)\geqslant 0, \text{ причем } (x,x) =0 \Leftrightarrow x=\theta\\ 
        &2^{\circ} \text{ коммутативность: } (y,x) = (x,y)\\ 
        &3^{\circ} \text{ линейность по первому аргументу: } (\alpha x + \beta y, z) = \alpha (x,z) + \beta (y,z)
    \end{aligned}$\\ 
    Сама функция $(,)$ называется скалярным произведением (СП), а свойства $1^{\circ}-3^{\circ}$ - аксиомами СП. 
\end{definition}
Замечание. Из $2^{\circ}$ и $3^{\circ}\implies$ линейность по втором аргументу, т.е $(x,\alpha y +\beta z ) = \alpha(x,y) + \beta(x,z)$

Примеры:
\noindent 1. ЛПВ(геом) $|\vec{x}| \cdot |\vec{y}|\neq  0 \quad (\vec{x},\vec{y}) \underset{\text{def}}{=} \begin{cases}
    |\vec{x}||\vec{y}| \cos{( \vec{x}{,}^{\wedge}\vec{y})} \\ 
    0, |\vec{x}||\vec{y}| = 0 
\end{cases}$ \\ 
\noindent 2. $\R^{n}$ (вещественные строки или столбцы длины или высоты $n$) \qquad $x=(\xi_{1}, \dots ,\xi_{n}), y = (\eta_{1}, \dots ,\eta_{n})$\\
$(x,y)\overset{\text{def}}{=} \xi_{1}\eta_{1} + \xi_{2}\eta_{2} + \dots + \xi_{n}\eta_{n}$, $1^{\circ}-3^{\circ}$ - очевидно (самостоятельно)\\ 
\noindent 3. $C[a,b]$ - вещественные функции, непрерывные на $[a,b]\qquad (x(t),y(t))= \int\limits_{a   }^{b    } x(t)y(t)dt$, $1^{\circ}- 3^{\circ}$ - очевидно (самостоятельно)\\

\begin{definition}
    $\V$ - ЛП над $\C$ называется евклидовым (комплексным евклидовым пространством - КЕП), если в нем определена функция $(,): \V \times \V\to \C: \forall x,y ,z \in \V$ и $\forall \alpha,\beta \in \C$ выполнены следующие свойства:\\ 
    $\begin{aligned}
        &1^{\circ} (x,x)\geqslant 0, \text{ причем } (x,x) =0 \Leftrightarrow x=\theta\\ 
        &2^{\circ} \text{ коммутативность: } (y,x) = \overline{(x,y)}\\ 
        &3^{\circ} \text{ линейность по первому аргументу: } (\alpha x + \beta y, z) = \overline{\alpha} (x,z) + \overline{\beta} (y,z)
    \end{aligned}$\\ 
    Сама функция $(,)$ называется скалярным произведением (СП), а свойства $1^{\circ}-3^{\circ}$ - аксиомами СП. 
\end{definition}
Замечание. Из $2^{\circ}$ и $3^{\circ}\implies$ линейность по втором аргументу, т.е $(x,\alpha y +\beta z ) = \overline{\alpha}(x,y) + \overline{\beta}(x,z)$
\newpage
Примеры: \\
\noindent 1. $\C^{n}$ (комплексные строки или столбцы длины или высоты $n$) \qquad $x=(\xi_{1}, \dots ,\xi_{n}), y = (\eta_{1}, \dots ,\eta_{n})$\\
$(x,y)\overset{\text{def}}{=} \xi_{1}\overline{\eta_{1}} + \xi_{2}\overline{\eta_{2}} + \dots + \xi_{n}\overline{\eta_{n}}$, $1^{\circ}-3^{\circ}$ - очевидно (самостоятельно)\\ 

\noindent 2. $\C[a,b]$ - комплексно-значные функции вещественных переменных, непрерывные на $[a,b]\\ (x(t),y(t))= \int\limits_{a   }^{b    } x(t)\overline{y(t)}dt$, $1^{\circ}- 3^{\circ}$ - очевидно (самостоятельно)\\
\\ $\begin{aligned}
    &x(t) = e^{int} = \cos{nt} + i\sin{nt}, \quad \C[0,2\pi], m ,n \in \Z\\ 
    &y(t) = e^{imt} = \cos{mt} + i\sin{mt}
\end{aligned}\\ (x(t),y(t)) = \int\limits_{0}^{2\pi} e^{int}e^{-imt} dt = \begin{aligned}
    &m=n \; \int\limits_{0}^{2\pi} dt = 2\pi \\ 
    &m\neq  n \; \int\limits_{0}^{2pi} e^{i(n-m)t} dt = \frac{1}{i(n-m)}e^{i(nim)t}\bigg|_{0}^{2\pi} = 0
\end{aligned}$\\ 
Дальнейшее обозначение: $\begin{aligned}
    &\mathbb{E} - \text{ВЕП ($\equiv$ евклидово)} \\
    &\U - \text{КЕП ($\equiv$ унитарное)} 
\end{aligned}$ 

\section{Нормированное ЛП} 
Пусть $\V$ - ЛП над $k$
\begin{definition}
    $\V$ называется нормированным ЛП (НЛП), если в нем определена фукнция $\|.\|: \V\to \R$, причем $\forall x ,y \in \V \; \forall \alpha \in k $ выполнены:$
    \begin{aligned}
        &1^{\circ} \|x\| >0 , \text{ причем } \|x\| =0 \Leftrightarrow x=\theta\\
        &2^{\circ} \text{ Однородность: } \|\alpha x\| = |\alpha| \cdot \|x\|\\
        &3^{\circ} \text{ Неравенство треугольника: } \|x+y\| \leqslant \|x\| + \|y\|
    \end{aligned}$
    \\ Сама функция $\|.\|$ называется нормой, а свойства $1^{\circ}-3^{\circ}$ - аксиомами нормы.
\end{definition}
В $\E$ и $\U$ нормой называют число $\|x\| = \sqrt{(x,x)}$, свойства $1^{\circ}-2^{\circ}$ - очевидно (самостоятельно)
  
Примеры: \\ 
\noindent 1. ЛПВ (геом.)\; $\|\vec{x}\| = \sqrt{(\vec{x},\vec{x})}=|\vec{x}|$\\ 
\noindent 2. $\R^{n}\; \| x\| = \sqrt{\xi_{1}^{2}+\dots+\xi_{n}^{2}} \qquad \C^{n} \; \|x\| = \sqrt{ |\xi_{1}|^{2}+\dots+|\xi_{n}|^{2}} \\ $
\noindent 3. $C[a,b] \; \|x(t)\| = \sqrt{\int\limits_{a }^{b    } x^{2}(t)dt} \qquad \C [a,b] \; \|x(t)\| = \sqrt{\int\limits_{a    }^{b    } |x^{2}(t)|dt}$. В частности $\|e^{imt}\| = \sqrt{2\pi}$
\begin{lemma}
    Если $x\neq  \theta$, то $\left\|\frac{x}{\|x\|}\right\| = 1$
\end{lemma}
\begin{proof}
    $\left\| \frac{x}{\|x\|}\right\| ^{2} \overset{2^{\circ}}{=} \frac{1}{\|x\|}\|x\| = 1 $
\end{proof}
\begin{theorem}[Неравенство Коши-Буняковского]
    $\forall x,y \in \E ( \U) \implies |(x,y)| \leqslant \|x\| \cdot \|y\|$
\end{theorem}
\begin{proof}
    Для $\E$ (для $\U$ в файле)
    \\ Если $y=\theta$, то равенство выполнено. Пусть $y\neq \theta$. Рассмотрим $0\leqslant(x-\lambda y, x- \lambda y)= (x,x)-\\-\lambda (y,x)- \lambda(x,y) + \lambda^{2}(y,y) = \|x\|^{2}-2\lambda (x,y)+\lambda^{2} \underbrace{\|y\|^{2}}_{>0}$ - квадратный многочлен от $\lambda \implies\\\implies \frac{D}{4 } = (x,y)^{2}-\|x\|^{2}\|y\|^{2}\leqslant 0 \implies (x,y)^{2}\leqslant\|x\|^{2}\|y\|^{2}\implies |(x,y)| \leqslant \|x\|\|y\|$
\end{proof}
Замечание. $(x-\lambda y, x-\lambda y)= 0 \Leftrightarrow x-\lambda y = \theta$, т.е $\{x,y\}$ - ЛЗ, т.е равенство в неравенстве Коши-Буняковского достигается $\Leftrightarrow \{x,y\}$ - ЛЗ. 

Замечание. В евклидовом пространстве можно ввести понятие угла между ненулевыми векторами $x{,}^{\wedge} y: x{,}^{\wedge} y \overset{\text{def}}{=} \arccos{\frac{(x,y)}{\|x\|\|y\|}} $

Примеры. Реализация неравенства Коши-Буняковского в различных НЛП:

\noindent 1. ЛПВ $|(\vec{x},\vec{y})\leqslant |\vec{x}||\vec{y}|$ \\ 
\noindent 2. $\R^{n} \; (\xi_{1} \eta_{1} + \dots+ \xi_{n}\eta_{n})^{2}\leqslant (\xi_{1}^{2}+\dots+\xi_{n}^{2})(\eta_{1}^{2}+\dots+\eta_{n}^{2})$\\ 
\noindent $\C^{n} \; |\xi_{1} \overline{\eta_{1}}+ \dots+ \xi_{n}\overline{\eta_{n}}|^{2}\leqslant (|\xi_{1}|^{2}+\dots+|\xi_{n}|^{2})(|\eta_{1}|^{2}+\dots+|\eta_{n}|^{2})$\\ 
\noindent 3. $C[a,b]\; \left(\int\limits_{a }^{b    } x(t)y(t)dt\right)^{2}\leqslant \int\limits_{a }^{b    } x^{2}(t)dt \leqslant \int\limits_{a   }^{b    } y^{2}(t)dt$\\ 
\noindent $\C[a,b]\; \left| \int\limits_{a  }^{b    } x(t)\overline{y(t)}dt\right|^{2}\leqslant \int\limits_{a  }^{b    } |x(t)|^{2}dt\int\limits_{a    }^{b    } |y(t)|^{2}dt$
\\Докажем теперь $3^{\circ}$ (неравенство треугольника) 
\begin{proof}
    В $\E$ (для $\U$ см. файл) \\ 
    $\|x+y\|^{2} = (x+y,x+y) = (x,x) +2(x,y)+(y,y)\leqslant \|x\|^{2}+2|(x,y)|+\|y\|^{2}\overset{\text{К-Б}}{\leqslant} \|x\|^{2}+2\|x\|\|y\|+\|y\|^{2}=\\=\underbrace{(\|x\|+\|y\|)^{2}}_{\geqslant 0}\implies \|x+y\| \leqslant \|x\| + \|y\| $
\end{proof}

\section{Общий вид СП}
Пусть $\V = \E , \dim{V} = n, \mathcal{E}=\{e_{1},\dots,e_{n}\}$ - некоторый базис в $\V$, $\begin{aligned}
&x=[\mathcal{E}]\underset{\downarrow}{\xi} = \sum_{i    =1}^{n  } \xi_{i}e_{i} \\ 
&y = [\mathcal{E}]\underset{\downarrow}{\eta}=\sum_{j=1}^{n } \eta_{j}e_{j}
\end{aligned}$\\ 
$(x,y) = \left(\sum_{i=1}^{n} \xi_{i}e_{i}, \sum_{j=1}^{n   } \eta_{j}e_{j} \right) = \sum_{i   =1}^{n  } \xi_{i}(e_{i},\sum_{j=1}^{n   } \eta_{j}e_{j}) = \sum_{i  =1}^{n} \sum_{j=1}^{n   } \xi_{i}\eta_{j}\underbrace{(e_{i},e_{j})}_{\gamma_{ij}}$ \qquad $\gamma_{ij} = (e_{i},e_{j}) = (e_{j},e_{i}) = \gamma_{ji}$
\begin{definition} 
    $\text{Г}^{\mathcal{E}} = (\gamma_{ij})^{n}_{n}$ называется матрицей Грама базиса $\mathcal{E}$
\end{definition}

Общий вид СП: $(x,y) = \sum_{i  =1}^{n}\sum_{j=1}^{n} \xi_{i}\eta_{j}\gamma_{ij} = \vec{\xi}\text{Г}^{\mathcal{E}}\underset{\downarrow}{\eta}$ - симметрическая БФ. \\ 
\noindent Какими свойствами должна обладать БФ $B(x,y)$, чтобы она порождала СП?\\ 
Любая БФ, удовлетворяющая свойству $3^{\circ}$ (и линейности по второму аргументу) свойства $2^{\circ}$ выполнены $\Leftrightarrow $ БФ симметрическая. \\ 
Свойство $1^{\circ}$ выполнено $\Leftrightarrow $ симметрическая БФ $B(x, y)$ порождает положительно определенную КФ $\Leftrightarrow$ матрица симметрической БФ имеет положительные главные угловые миноры. \\ 
Итог: любая симметрическая БФ, порождающая положительно определенную КФ, порождает в вещественном ЛП скалярное произведение $\implies \text{Г}^{\mathcal{E}}$ симметрическая и ее главные угловые миноры $>0$.
 
Пусть $\V = \U , \dim{V} = n, \mathcal{E}=\{e_{1},\dots,e_{n}\}$ - некоторый базис в $\V$, $\begin{aligned}
    &x=[\mathcal{E}]\underset{\downarrow}{\xi} = \sum_{i    =1}^{n  } \xi_{i}e_{i} \\ 
    &y = [\mathcal{E}]\underset{\downarrow}{\eta}=\sum_{j=1}^{n } \eta_{j}e_{j}
    \end{aligned}$\\ 
    $(x,y) = \left(\sum_{i=1}^{n} \xi_{i}e_{i}, \sum_{j=1}^{n   } \eta_{j}e_{j} \right) = \sum_{i   =1}^{n  } \xi_{i}(e_{i},\sum_{j=1}^{n   } \eta_{j}e_{j}) = \sum_{i  =1}^{n} \sum_{j=1}^{n   } \xi_{i}\overline{\eta_{j}}\underbrace{(e_{i},e_{j})}_{\gamma_{ij}}$ \qquad $\gamma_{ij} = (e_{i},e_{j}) = \overline{(e_{j},e_{i})} = \overline{\gamma_{ji}}$

    Общий вид СП в $\U$: $(x,y) = \sum_{i  =1}^{n}\sum_{j=1}^{n} \xi_{i}\overline{\eta_{j}}\gamma_{ij} = \vec{\xi}\text{Г}^{\mathcal{E}}\underset{\downarrow}{\overline{\eta}}$ - эрмитова ПФ. \\ 
    \noindent Какими свойствами должна обладать ПФ $B(x,y)$, чтобы она порождала СП в комплексном ЛП?\\ 
    Любая ПФ, удовлетворяющая свойству $3^{\circ}$ (и линейности по второму аргументу) свойства $2^{\circ}$ выполнены $\Leftrightarrow $ ПФ эрмитова (эрм. симм.) \; $M=\begin{pmatrix}
        1 & 1+i \\ 
        1-i & 3 
    \end{pmatrix}$ - эрмитово симм. матрица $(M^{T}=\overline{M}\Leftrightarrow M^{*}=M)$\\ 
    Свойство $1^{\circ}$ выполнено $\Leftrightarrow $ эрмитова ПФ $B(x, y)$ порождает положительно определенную КФ $\Leftrightarrow$ матрица эрмитово ПФ имеет положительные главные угловые миноры. \\ 
    Итог: любая эрмитово ПФ, порождающая положительно определенную КФ, порождает в комплексном ЛП скалярное произведение $\implies \text{Г}^{\mathcal{E}}$ эрмитова и ее главные угловые миноры $>0$.
     

    \section{ОНБ В $\E(\U)$}
Пусть $\V= \E $ (или $\U$)
\begin{definition}
    $x,y\in \V$ называются ортогональными, если $(x,y)= 0$. Обозначение: $x\perp y$
\end{definition}
\begin{definition}
    Базис $\mathcal{E} = \{e_{1},\dots,e_{n}\}$ называется ортогональным (ОБ), если $\forall i,j=\overline{1,n}: i\neq j\implies\\\implies (e_{i},e_{j})=0$
    
\end{definition}
\begin{definition}
    Базис $\mathcal{E}$ называется ортонормированным (ОНБ), если $\forall i,j =\overline{1,n}\implies (e_{i},e_{j})=\delta_{ij}$ 
\end{definition}
\begin{definition}
    Система векторов $\{a_{1},\dots,a_{m}\}$ называется ортогональной (ОС), если $\forall i,j=\overline{1,m}: \\i\neq j \implies (a_{i},a_{j})=0$ 

\end{definition}
\begin{definition}
    Система векторов $\{a_{1},\dots,a_{m}\}$ называется ортонормированной (ОНС), если $\forall i,j=\overline{1,m}\implies (a_{i},a_{j})=\delta_{ij}$ 

\end{definition}


Замечание. ОС = ОБ, если она является базисом. ОНС = ОНБ, если она является базисом. 

Замечание. $\text{Г}_{\text{ОБ}}= \Lambda = \begin{pmatrix}
    \lambda_{1}& 0 &\dots&0 \\ 
    0& \lambda_{2} & \dots & 0 \\ 
    \vdots & \vdots & \ddots & \vdots \\ 
    0 & 0 & \dots & \lambda_{n}
\end{pmatrix}$ \qquad $\forall i=\overline{1,n}\implies \lambda_{i}>0$ \\ 
$\text{Г}_{\text{ОНБ}} = \begin{pmatrix}
    1& 0 &\dots&0 \\ 
    0& 1 & \dots & 0 \\ 
    \vdots & \vdots & \ddots & \vdots \\ 
    0 & 0 & \dots & 1
\end{pmatrix}=E\implies $ вид СП в ОБ: 
$\begin{aligned}
 &\tcircle{$\E$} \;(x,y) = \sum_{i =1}^{n  } \lambda_{i} \xi_{i}\eta_{i} \\
 &\tcircle{$\U$} \;(x,y) = \sum_{i =1}^{n} \lambda_{i}\xi_{i}\overline{\eta_{i}}  
\end{aligned}$
\noindent Вид СП в ОНБ:
$\begin{aligned}
 &\tcircle{$\E$}\; (x,y) = \sum_{i =1}^{n  } \xi_{i}\eta_{i} \\
 &\tcircle{$\U$} \;(x,y) = \sum_{i =1}^{n} \xi_{i}\overline{\eta_{i}}  
\end{aligned}$



\vspace{1cm}
\begin{flushright}
    \textit{tg: @moksimqa}
\end{flushright}