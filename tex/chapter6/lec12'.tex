\documentclass[../main.tex]{subfiles}
\begin{document}
\lecture{12}{21.04}{}

\section{Определение и общий вид ПФ в комплексном ЛП}
Пусть $\V$ - ЛП над $\C$
\begin{definition}
    Закон (правило, функция) $B : \V \times \V \to \C$ (упорядоченная пара элементов $x,y\in\V$ ставит в соответствие комплексное число) называется полуторалинейной формой (ПФ), если $\forall x,y,z\in\V $ и $\forall \alpha \in \C \implies $
    $
    \begin{aligned}
        &1^{\circ} B(\alpha x + \beta y, z)= \alpha B(x,z) + \beta B(y,z) \\ 
        &2^{\circ} B(x,\alpha y + \beta z ) = \overline{\alpha} B(x,y) + \overline{\beta} B(x,z) \;(\overline{\alpha},\overline{\beta} - \text{ комплексно сопряженные числа к $\alpha,\beta$})
    \end{aligned}$
\end{definition}

Пусть $\mathcal{E} = \{ e_{1},\dots,e_{n}\}$ - базис в $\V, x = [\mathcal{E}]\underset{\downarrow}{\xi}, y = [\mathcal{E}]\underset{\downarrow}{\eta}$. 
Тогда $B(x,y) = \sum_{i =1}^{n  } \sum_{j=1}^{n} \xi_{i}\overline{\eta_{j}} \underbrace{B(e_{i},e_{j})}_{b_{ij}} =\\= \sum_{i  =1}^{n  } \sum_{j=1}^{n } b_{ij}\xi_{i}\overline{\eta_{j}} $ - общий вид ПФ 
\section{Матрица ПФ. Преобразование матрицы ПФ при смене базиса}
\begin{definition}
    $M_{B}^{\mathcal{E}} = (b_{ij})^{n}_{n} = \begin{pmatrix}
        B(e_{1},e_{1}) & \dots & B(e_{1},e)_{n} \\ 
        \vdots & \ddots & \vdots \\ 
        B(e_{n},e_{1}) & \dots & B(e_{n},e_{n})
    \end{pmatrix}$ называется матрицей ПФ $B$ в базисе $\mathcal{E}$
\end{definition}
С помощью нее получаем, что $B(x,y) = \vec{\xi} M_{B}^{\mathcal{E}}\overline{\underset{\downarrow}{\eta}}$
\begin{theorem}
    Если $[\mathcal{E'}] = [\mathcal{E}] T_{\mathcal{E}\to\mathcal{E'}}, \mathcal{E'} = \{e_{1}',\dots,e_{n}'\}$ - другой базис в $\V$, то $M_{B}^{\mathcal{E'}} = T_{\mathcal{E}\to\mathcal{E'}}^{t}M_{B}^{\mathcal{E}}T_{\mathcal{E}\to\mathcal{E'}}$
\end{theorem}
\begin{proof}
    См. файл.
\end{proof}

Существует ВОС между ПФ и квадратными комплекснозначными матрицами. 
\begin{proof}
    См. файл
\end{proof}
    
\begin{definition}
    $B_{1}=B_{2}$, если $\forall x,y \in \V\implies B_{1}(x,y)= B_{2}(x,y)$
\end{definition}

\begin{definition}
    ПФ называется эрмитовыой, если $\forall x,y \in \V\implies B(x,y)=\overline{B(y,x)}$
\end{definition}
\begin{definition}
    $M= (m_{ij})^{n}_{n}$ называется эрмитовой (эрмитово симметричной), если $M^{T}= \overline{M}$
\end{definition}
\vspace{0.2cm}
Пример: $\begin{pmatrix}
    1 & 1+i \\ 
    1-i & 2
\end{pmatrix}^{T} = \begin{pmatrix}
    1 & 1-i \\ 
    1+i & 2

\end{pmatrix} = \overline{\begin{pmatrix}
    1 & 1+i \\ 
    1-i & 2
\end{pmatrix}}$


\begin{theorem}
    ПФ $B$ - эрмитова $\Leftrightarrow $ в некотором базисе ее матрица является эрмитово симметричной.    
\end{theorem}
\begin{corollary}
    У эрмитовой ПФ в любом базисе матрица является эрмитово симметричной.    
\end{corollary}

\vspace{1cm}
\begin{flushright}
    \textit{tg: @moksimqa}
\end{flushright}