\documentclass[../main.tex]{subfiles}
\begin{document}
\lecture{4}{3.03}{}
Пример $\begin{cases}
    x_{1}+x_{2}+x_{3}+x_{4}=4 \\ 
    x_{1}+x_{2}+\phantom{x_{3}+x_{4}}=2
\end{cases}$
\\ $\begin{pmatrix}[cccc|c]
    1 & 1 & 1 & 1 & 4 \\ 
    1 & 1 & 0 & 0 & 2
\end{pmatrix}\sim \begin{pmatrix}[cccc|c]
    1 & 1 & 1 & 1 & 4 \\ 
    0 & 0 & -1 & -1 & -2
\end{pmatrix}\sim \begin{pmatrix}[cccc|c]
    x_{1}& x_{3}& x_{2}&x_{4} \\ 
    1& 1 & 1 & 1 &4 \\
    0 & -1 & 0 & -1 & -2
\end{pmatrix}\sim \begin{pmatrix}[cc|cc|c]
    1 & 0  & 1 & 0 & 2 \\
    0 & 1 & 0 & 1 & 2
\end{pmatrix}\\ RgA=2, n = 4 $
\\ Исходная СЛАУ$\sim\begin{cases}
    x_{1}=2-x_{2}\\ 
    x_{3}=2-x_{4}
\end{cases} \begin{aligned}\qquad &\text{Общее решение в координатной форме:} \begin{cases}
    x_{1} = 2-C_{1} \\ 
    x_{3} = 2 - C_{2} \\
    x_{2}= C_{1} \\ 
    x_{4} = C_{2} 
\end{cases} \\ &\text{В векторной форме: } \underset{\downarrow}{x}=\begin{pmatrix}
    2 -C_{1} \\ 
    C_{1} \\ 
    2- C_{2}\\ 
    C_{2}
\end{pmatrix}\; C_{1},C_{2},C_{3} \in \R \end{aligned}$


\section{Теорема Кронекера-Капелли.}
\begin{theorem}[Критейрий совместности СЛАУ]
    СЛАУ (1) $A\underset{\downarrow}{x}=\underset{\downarrow}{b}$ совместна $\Leftrightarrow RgA = Rg(A|\underset{\downarrow}{b})$
    \\($Rg(\underset{\downarrow}{a_{1}},\dots,\underset{\downarrow}{a_{n}})=Rg(\underset{\downarrow}{a_{1}},\dots,\underset{\downarrow}{a_{n}}|\underset{\downarrow}{b})$)

\end{theorem}
\begin{proof}
    СЛАУ (1) можно записать в виде эквивалентой форме: $(2) \; x_{1} \underset{\downarrow}{a_{1}}+\dots+x_{n}\underset{\downarrow}{a_{n}}=\underset{\downarrow}{b}$    
    \\ $\tcircle{$\implies$}$ Пусть СЛАУ (1) совместна $\implies \exists x_{1},\dots,x_{n}: $ выполнено $(2)\implies\underset{\downarrow}{b}$ является ЛК $\underset{\downarrow}{a_{1}},\dots,\underset{\downarrow}{a_{n}} \implies \underset{\downarrow}{b}$ линейно зависит от $\underset{\downarrow}{a_{1}},\dots,\underset{\downarrow}{a_{n}}\implies $ число ЛНЗ столбцов в системах $\{\underset{\downarrow}{a_{1}},\dots,\underset{\downarrow}{a_{n}}\}$ и $\{\underset{\downarrow}{a_{1}}.\dots,\underset{\downarrow}{a_{n}|\underset{\downarrow}{b}}\} $ одинаковое $\implies RgA=Rg(A|\underset{\downarrow}{b})$
    \\ $\tcircle{$\impliedby$}$ \;$RgA=Rg(A|\underset{\downarrow}{b})=r\implies$ в области матрицы $A\; \exists \tcircle{$M_{r}$}\neq 0$, его столбцы ЛНЗ и любые столбцы матрицы $A$ являются ЛК столбцов, входящих в этот базисный минор $\implies \underset{\downarrow}{b}=\alpha_{1}\underset{\downarrow}{a}_{i1}+\dots+a_{r}\underset{\downarrow}{a}_{ir}\implies$ (2) имеет решение $\implies $ (1) совместна.  
\end{proof}

\newpage \section{Однородные СЛАУ.}
\begin{definition}
    СЛАУ $A \underset{\downarrow}{x}= \underset{\downarrow}{b}$ называется \underline{однородной}, если $\underset{\downarrow}{b}=\underset{\downarrow}{0}$, т.е $A \underset{\downarrow}{x}= \underset{\downarrow}{0}$ ($1_{0}$) - ОСЛАУ
\end{definition}
Замечание. ОСЛАУ всегда совместна. Ее решение $\underset{\downarrow}{x}=\underset{\downarrow}{0}$ называется \underline{тривиальным}. Прочие решения, если они имеются, называются \underline{нетривиальными}.
\begin{theorem}[О ЛК решений ОСЛАУ]
    Если $\underset{\downarrow}{x}^{(1)},\dots, \underset{\downarrow}{x}^{(k)}$ - любые решение ОСЛАУ, то $\forall C_{1},\dots,C_{k}\in \\ \in \R(\C)\implies C_{1}\underset{\downarrow}{x}^{(1)}+\dots+C_{k}\underset{\downarrow}{x}^{(k)}=\underset{\downarrow}{\alpha}$ - тоже решение этой ОСЛАУ.
\end{theorem}
\begin{proof}
    $A\underset{\downarrow}{a}=A(C_{1}\underset{\downarrow}{x}^{(1)}+\dots+C_{k}\underset{\downarrow}{x}^{(k)})=C_{1}A\underset{\downarrow}{x}^{(1)}+\dots C_{k}A \underset{\downarrow}{x}^{(k)}= C_{1}\underset{\downarrow}\cdot{0}+\dots C_{k}\cdot\underset{\downarrow}{0}=\underset{\downarrow}{0}$
\end{proof}
\begin{corollary}
    Если ОСЛАУ имеет хотя бы одно нетривиальное решение, то их будет бесконечно много.
\end{corollary}
\begin{theorem}
    $\begin{aligned} &1) \text{Если } RgA=r=n \;\text{(число неизвестных)},\text{ то ОСЛАУ обладает только тривиальным решением. }\\ &2) \text{Если } RgA=r<n, \text{ то ОСЛАУ имеет нетривиальные решения. }  \end{aligned}$
\end{theorem}
\begin{proof}
    $\tcircle{$2$}\; \begin{pmatrix}[ccc|c]
    a_{11}&\dots & a_{1n} & b_{1}\\ 
    \vdots & \ddots & \vdots & \vdots \\ 
    a_{m1} & \dots & a_{mn} & b_{m}
\end{pmatrix}(1_{0})\sim \text{эл. преобр. только строк }\sim \begin{pmatrix}[ccccc|c]
    a'_{11} & \dots & a'_{1r}&\dots  & a'_{1n}& 0 \\
    0 & \ddots & \dots &\dots  & a'_{1n}& \vdots \\
    0 & \dots & a'_{rr} & \dots& a'_{rn} &0  \\ \hline
    \multicolumn{4}{c}{\text{НУЛИ}}& &0\\\hline
    \multicolumn{4}{c}{\text{НУЛИ}}& &0
\end{pmatrix}\;(2_{0}) \\ x_{1},\dots,x_{r} - \text{ главные}, x_{r+1},\dots,x_{n}-\text{ свободные } \begin{cases}
    x_{1}=-a''_{1r+1}x_{r+1}-\dots-a''_{1n}x_{n} \\ 
    \phantom{x_{1}=-a''_{1r+1}x} \dots \\ 
    x_{r}=-a''_{rr+1}x_{r+1}-\dots-a''_{rn}x_{n} \\ 
    x_{r+1},x_{r+2},\dots,x_{n} \in\mathbb{R}
\end{cases}$ \\Методом Г-Ж ($1_{0}$) приводится к $(2_{0})$, причем $n-r>0\implies$ свободные переменные имеются.  \\Положим $x_{r+1}=x_{n-1}=0,x_{n}=1$.
Тогда получим $\underset{\downarrow}{\alpha}=\begin{pmatrix}
     -a''_{1n}\\ 
     \vdots \\ 
     -a''_{rn}\\
     0\\ 
     \vdots \\ 
    0\\ 
    1
\end{pmatrix}$ - нетривиальное решение.\\ 
$\tcircle{$2$}\;\;$Если $r=n$, то в $(2_{0})$ не будет свободных переменных: $\begin{pmatrix}[cccc|c]
    1 & 0 & \dots & 0 & 0\\
    0 & 1 & \dots & 0 &0\\
    \vdots & \vdots & \ddots & \vdots & 0\\
    0 & 0 & \dots & 1 & 0 \\ 
    \hline 
    0 & 0 & \dots & 0 & 0
    
\end{pmatrix},$ т.е $(1_{0})\sim (3_{0}) \; \begin{cases}
    x_{1}=0 \\ 
    \vdots \\ 
    x_{n} =0
\end{cases} $ т.е имеется только тривиальное решение.

\end{proof}

\section{Фундаментальная система решений ОСЛАУ.}
Рассмотрим ОСЛАУ. $(1_{0}) \; A\underset{\downarrow}{x}=\underset{\downarrow}{0}$
\begin{definition}
    Упорядоченная, ЛНЗ система этой ОСЛАУ $(1_{0})$ $\underset{\downarrow}{\varphi}^{(1)},\dots,\underset{\downarrow}{\varphi}^{(k)}$ называется фундаментальной системой решений (ФСР), если для любого решения ОСЛАУ $(1_{0})\implies \exists C_{1},\dots,C_{k}\in \R(\C): \underset{\downarrow}{\alpha}=\\=C_{1}\underset{\downarrow}{\varphi}^{(1)}+\dots+C_{k}\underset{\downarrow}{\varphi}^{(k)}$
\end{definition} 
\begin{theorem}[О нормальной системе решений (НСР)]
    Если $RgA=r<n$, то ОСЛАУ $(1_{0})$ имеет $(n-r)$ ЛНЗ решений, через которые выражаются любое решения.
    
\end{theorem}
\begin{proof}
    Пусть $RgA=r<n$ (число неизвестных). Тогда $(1_{0})\sim(2_{0})$ с $(n-r)$ свободных переменных, которым придадим следующие наборы значений:$\begin{aligned} &x_{r+1}=1, x_{r+2}=0,\dots,x_{n}=0 \\ &x_{r+1}=0,x_{r+2}=1,\dots,x_{n}=0\\&\phantom{x_{r+1}=0,x_{r+2}}\dots\\ &x_{r+1}=0,x_{r+2}=0,\dots,x_{n}=1\end{aligned}$\\По этим наборам найдем значения главных переменных, получим $(n-r)$ решений: \\ $\underset{\downarrow}{\varphi}^{(1)}=\begin{pmatrix}
        -a''_{1r+1}\\
        \vdots\\
        -a''_{rr+1}\\
        1\\ 
        0\\ 
        \vdots\\ 
        0
    \end{pmatrix}\; \underset{\downarrow}{\varphi}^{(2)}=\begin{pmatrix}
        -a''_{1r+2}\\
        \vdots\\
        -a''_{rr+2}\\
        0\\ 
        1\\ 
        \vdots\\ 
        0
    \end{pmatrix}\;\dots\; \underset{\downarrow}{\varphi}^{(n-r)} = 
    \begin{pmatrix}
        -a''_{1n}\\
        \vdots\\
        -a''_{rn}\\
        0\\ 
        0\\ 
        \vdots\\ 
        1
    \end{pmatrix}$ 
    \\ Рассмотрим $\varPhi = (\underset{\downarrow}{\varphi}^{(1)},\dots,\underset{\downarrow}{\varphi}^{(n-r)})=r\begin{pmatrix}
        -a''_{1r+1} & -a''_{1r+2} & \dots & -a''_{1n}\\
        \vdots & \vdots & \ddots & \vdots\\
        -a''_{rr+1} & -a''_{rr+2} & \dots & -a''_{rn}\\
        1 & 0 & \dots & 0\\
        0 & 1 & \dots & 0\\
        \vdots & \vdots & \ddots & \vdots\\
        0 & 0 & \dots & 1
    \end{pmatrix}$ \\В нижней части $\varPhi$ имеется $\tcircle{$M_{n-r}$}=detE=1\neq 0$, у $\varPhi$ $(n-r)$ столбцов $\implies Rg\varPhi = n-r\implies$ ее столбцы ЛНЗ $\implies \underset{\downarrow}{\varphi}^{(1)},\dots,\underset{\downarrow}{\varphi}^{(n-r)}$ - ЛНЗ система решений 
\end{proof}
\begin{definition}
    Построенная таким образом система решений называется \underline{нормальной} (НСР)
\end{definition}
\newpage Покажем теперь, что $\forall \underset{\downarrow}{\alpha} - $ решение $(1_{0})$ можно представить в виде ЛК $\underset{\downarrow}{\varphi}^{(1)},\dots,\underset{\downarrow}{\varphi}^{(n-r)}$. 
Пусть $\underset{\downarrow}{\alpha}=\begin{pmatrix} \vec{\alpha}_{1}\\ \vdots\\ \vec{\alpha}_{n} \end{pmatrix}$ - любое решение $(1_{0})$. 
Рассмотрим $\underset{\downarrow}{y}=\underset{\downarrow}{\alpha_{1}}-\alpha_{r+1}\underset{\downarrow}{\varphi}^{(1)}+\dots+\alpha_{n}\underset{\downarrow}{\varphi}^{(n-r)}=\begin{pmatrix}
    \beta_{1} \\ 
    \vdots \\ 
    \beta_{r} \\ 
    0 \\ 
    \vdots \\ 
    0
\end{pmatrix}$ \\Поскольку $\underset{\downarrow}{y}$ является ЛК решений ОСЛАУ $(1_{0})$, то $\underset{\downarrow}{y}$ тоже является решением $(1_{0})\implies$ его компаненты удовлетворяют $(2_{0})$. Откуда, учитывая, что все свободные переменные равны 0, получим (см. $(2_{0})$) $\begin{cases}
    \beta_{1}=0 \\ 
    \vdots \\ 
    \beta_{r}=0
\end{cases},$ т.е $\underset{\downarrow}{y}=\underset{\downarrow}{0}\implies \underset{\downarrow}{\alpha}=\alpha_{r+1}\underset{\downarrow}{\varphi}^{(1)}+\dots+\alpha_{n}\underset{\downarrow}{\varphi}^{(n-r)}$. Заметим, что отсюда следует, что б$\acute{\text{о}}$льшего, чем $(n-r)$ количества ЛНЗ решений быть не может.

\begin{theorem}[О ФСР]
    Если $RgA=r=n,$ то ФСР ОСЛАУ не существует, если $RgA=r<n,$ то\\ $\begin{aligned}
        1) &\exists \text{ ФСР ОСЛАУ } (1_{0}) \\ 
        2) &\text{Любая ФСР ОСЛАУ } (1_{0}) \text{ содержит ровно } (n-r) \text{ элементов }\\ 
        3) &\text{Любые } (n-r) \text{ ЛНЗ решений ОСЛАУ } (1_{0}) \text{ образуют ее ФСР} \\ 
        4) &\text{Если } \underset{\downarrow}{\varphi}^{(1)},\dots,\underset{\downarrow}{\varphi}^{(n-r)} - \text{ некоторая ФСР ОСЛАУ } (1_{0}), \text{ то ее общее решение имеет вид: }\\  &\underset{\downarrow}{x_{\text{оо}}}=C_{1}\underset{\downarrow}{\varphi}^{(1)}+\dots+C_{n-r}\underset{\downarrow}{\varphi}^{(n-r)}, \text{ где } C_{1},\dots,C_{n-r} - \text{ произвольные числа } \R (\C)
    \end{aligned}$

\end{theorem}
\begin{proof}
    $RgA=r=n\implies$ имеется только тривиальное решение (оно всегда ЛЗ) $\implies$ нет ФСР.  
    \\ $RgA=r<n \implies$ 
    \\1) Уже доказано, т.к $\exists$ НСР, она является частным случаем ФСР. 
    \\2) Будет доказано позже. 
    \\3) Будет доказано позже. 
    \\4) $\tcircle{$\impliedby$}$ Поскольку $\underset{\downarrow}{\varphi}^{(1)},\dots,\underset{\downarrow}{\varphi}^{(n)}$ - решение ОСЛАУ, то любая их ЛК также является решением (см. выше).
    \\ $\tcircle{$\implies$}$ Пусть $\underset{\downarrow}{y} $ - произвольное решение ОСЛУ. По определению ФСР $\exists \tilde{C}_{1},\dots,\tilde{C}_{n}: \underset{\downarrow}{y}=\tilde{C}_{1}\underset{\downarrow}{\varphi}^{(1)}+\dots+\tilde{C}_{n-r}\underset{\downarrow}{\varphi}^{(n-r)}$
\end{proof}



\vspace{1cm}
\begin{flushright}
    \textit{tg: @moksimqa}
\end{flushright}
\end{document}