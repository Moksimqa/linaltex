% ----------------------------------------------------
%    Основные настройки: кодировка, язык и шрифты
% ----------------------------------------------------
\usepackage[utf8]{inputenc}
\usepackage[T2A]{fontenc}
\usepackage{textcomp}
\usepackage[russian]{babel}

% ----------------------------------------------------
%    Пакеты для форматирования ссылок и гиперссылок
% ----------------------------------------------------
\usepackage{url}
\usepackage{hyperref}
\hypersetup{
    colorlinks=true,    % Цветные ссылки
    linkcolor=blue,     % Цвет внутренних ссылок
    urlcolor=cyan,      % Цвет внешних ссылок
    pdftitle={Линейная алгебра},    % Назваание PDF
    pdfauthor={Белуосов М.},        % Автор документа
    bookmarksopen=true,             % Открытие оглавления
    unicode=true                    % Кодировка для юникода
}


% ----------------------------------------------------
%    Работа с графикой и рисунками
% ----------------------------------------------------
\usepackage{graphicx}   % Вставка изображений
\usepackage{float}      % Позиционирование изображений
\usepackage{tikz}       % Векторная графика
%\usepackage{pgfplots}  % Построение графиков

% ----------------------------------------------------
%    Таблицы и списки
% ----------------------------------------------------
\usepackage{booktabs}    % Красивые таблицы
\usepackage{enumitem}    % Расширенные списки
\usepackage{multirow}    % Многострочные ячейки в таблицах

% ----------------------------------------------------
%    Дополнительные пакеты
% ----------------------------------------------------
\usepackage{epigraph}    % Эпиграфы
\usepackage{subfiles}    % Подключение отдельных файлов
\usepackage{titling}     % Настройка заголовков
\usepackage{fancybox}    % Рамки
\usepackage{indentfirst} % Красная строка

% ----------------------------------------------------
%    Геометрия и макет страницы
% ----------------------------------------------------
\usepackage[left=2cm,right=2cm,top=2cm,bottom=3cm,bindingoffset=0cm]{geometry}
\pdfminorversion=7
\usepackage{parskip} % Интервал между абзацами
\usepackage{emptypage} % Пустые страницы без номера
\usepackage{subcaption} % Подписи к рисункам
\usepackage{multicol} % Много колонок
\usepackage{xcolor} % Цвета

% ----------------------------------------------------
%    Математические пакеты
% ----------------------------------------------------
\usepackage{amsmath, amsfonts, mathtools, amsthm, amssymb}
\usepackage{mathrsfs}   % Скриптовые шрифты
\usepackage{cancel}     % Зачеркивание
\usepackage{bm}         % Жирные символы 
\usepackage{systeme}    % Системы уравнений
\everymath{\displaystyle} % Всегда использовать \displaystyle в формулах


% ----------------------------------------------------
%    Собственные обозначения и команды
% ----------------------------------------------------
\newcommand\N{\ensuremath{\mathbb{N}}} 
\newcommand\R{\ensuremath{\mathbb{R}}}
\newcommand\Z{\ensuremath{\mathbb{Z}}}
\renewcommand\O{\ensuremath{\emptyset}}
\newcommand\Q{\ensuremath{\mathbb{Q}}}
\renewcommand\C{\ensuremath{\mathbb{C}}}
\newcommand\V{\ensuremath{\mathbb{V}}}
\newcommand\W{\ensuremath{\mathbb{W}}}
\renewcommand\U{\ensuremath{\mathbb{U}}}
\newcommand\E{\ensuremath{\mathbb{E}}}
\newcommand*\tcircle[1]{%
  \tikz[baseline=(C.base)]\node[draw,circle,inner sep=0.5pt](C) {#1};\!
} % Кружок вокруг символа

% Для форматирования матриц (черта между столбцами)
\makeatletter
\renewcommand*\env@matrix[1][*\c@MaxMatrixCols c]{%
  \hskip -\arraycolsep
  \let\@ifnextchar\new@ifnextchar
  \array{#1}}
\makeatother

% ----------------------------------------------------
%    Математические операторы и команды
% ----------------------------------------------------
\newcommand{\defect}{\operatorname{def}} % Дефект
\DeclareMathOperator{\spann}{span}  % Оболочка
\renewcommand{\Im}{\operatorname{Im}}   % Образ
\newcommand\spanset[1]{\ensuremath\spann(#1)} 

\let\svlim\lim\def\lim{\svlim\limits} 

% Упрощение написания математических символов
\let\implies\Rightarrow
\let\impliedby\Leftarrow
\let\iff\Leftrightarrow
\let\epsilon\varepsilon

% Знак противоречия
\usepackage{stmaryrd} % for \lightning
\newcommand\contra{\scalebox{1.5}{$\lightning$}}

% \let\phi\varphi

% Команда \correct для исправления ошибок в примерах
% Usage: 1+1=\correct{3}{2}

\definecolor{correct}{HTML}{009900}
\newcommand\correct[2]{\ensuremath{\:}{\color{red}{#1}}\ensuremath{\to }{\color{correct}{#2}}\ensuremath{\:}}
\newcommand\green[1]{{\color{correct}{#1}}}

% Горизонтальная линия
\newcommand\hr{
    \noindent\rule[0.5ex]{\linewidth}{0.5pt}
}

% Пробелы в формуле
\newcommand\hide[1]{}

% ----------------------------------------------------
%    Окружения для теорем, определений и т.д.
% ----------------------------------------------------
\makeatother % 
% For box around Definition, Theorem, \ldots
\usepackage{mdframed} 
\mdfsetup{skipabove=1em,skipbelow=0em}
\theoremstyle{plain}
\newtheorem{theorem}{Теорема}[chapter]
\newtheorem{lemma}[theorem]{Лемма}
\newcounter{corollaryCounter}[theorem]
\newtheorem{corollary}[corollaryCounter]{Следствие}
\newtheorem*{corollary*}{Следствие}

\theoremstyle{definition}
\newtheorem*{example}{Пример}
\newtheorem*{examples}{Примеры}
\newtheorem*{iexample}{Важный пример}
\newtheorem*{exercise}{Упражнение}
\newtheorem*{definition}{Определение}

\theoremstyle{remark}
\newtheorem*{remark}{Замечание}
\newtheorem*{editremark}{Замечание (от редакторов конспекта)}
\newtheorem*{remarks}{Замечания}
\newtheorem*{exerciseAnswer}{Ответ на упражнение}


% Упрощённые обозначения
\let \thm \theorem
\let \lem \lemma
\let \defn \definition
\let \exmp \example
\let \iex \iexample
\let \exmps \examples
\let \exc \exercise
\let \rem \remark
\let \erem \editremark
\let \rems \remarks
\let \crl \corollary
\let \eans \exerciseAnswer



% Математические операторы
\DeclareMathOperator{\rank}{rank}
\DeclareMathOperator{\mes}{mes}
\DeclareMathOperator{\diam}{diam}
\DeclareMathOperator{\fix}{fix}
\DeclareMathOperator{\sgn}{sgn}
\DeclareMathOperator{\sign}{sgn}
\DeclareMathOperator{\vp}{v.p.}
\DeclareMathOperator{\Arg}{Arg}
\DeclareMathOperator{\Ln}{Ln}
\DeclareMathOperator{\Arcsin}{Arcsin}
\DeclareMathOperator{\Arccos}{Arccos}
\DeclareMathOperator{\Arctg}{Arctg}
\DeclareMathOperator{\Arcctg}{Arcctg}
\DeclareMathOperator{\Arsh}{Arsh}
\DeclareMathOperator{\Arch}{Arch}
\DeclareMathOperator{\Arth}{Arth}
\DeclareMathOperator{\Arcth}{Arcth}
\DeclareMathOperator{\res}{res}
\renewcommand{\arctan}{\arctg} % arctan -> arctg

% End example and intermezzo environments with a small diamond (just like proof
% environments end with a small square)
\usepackage{etoolbox}
\AtEndEnvironment{vb}{\null\hfill$\diamond$}%
\AtEndEnvironment{intermezzo}{\null\hfill$\diamond$}%
% \AtEndEnvironment{opmerking}{\null\hfill$\diamond$}%

% Фикс пробелов
\makeatletter
\def\thm@space@setup{%
  \thm@preskip=\parskip \thm@postskip=0pt
}


% Exercise 
% Usage:
% \oefening{5}
% \suboefening{1}
% \suboefening{2}
% \suboefening{3}
% gives
% Oefening 5
%   Oefening 5.1
%   Oefening 5.2
%   Oefening 5.3
\newcommand{\oefening}[1]{%
    \def\@oefening{#1}%
    \subsection*{Oefening #1}
}

\newcommand{\suboefening}[1]{%
    \subsubsection*{Oefening \@oefening.#1}
}


% \lecture начинает новую лекцию
%
% Использование:
% \lecture{1}{вт 12 фев 2019 16:00}{Введение}
%
% Эта команда добавляет заголовок раздела с номером/названием лекции и 
% примечание на полях с датой.

% Я использую \dateparts здесь, чтобы скрыть год. Таким образом, я могу 
% легко обрабатывать дату каждой лекции однозначно, при этом сохраняя 
% удобный для чтения короткий формат, отображаемый в PDF.

\usepackage{xifthen}
\def\testdateparts#1{\dateparts#1\relax}
\def\dateparts#1 #2 #3 #4 #5\relax{
    \marginpar{\small\textsf{\mbox{#1 #2 #3 #5}}}
}

\def\@lecture{}%
\newcommand{\lecture}[3]{
    \ifthenelse{\isempty{#3}}{%
        \def\@lecture{Лекция #1}%
    }{%
        \def\@lecture{Лекция #1: #3}%
    }%
    \subsection*{\@lecture}
    \marginpar{\small\textsf{\mbox{#2}}}
}


% ----------------------------------------------------
%    Колонтитулы
% ----------------------------------------------------
\usepackage{fancyhdr}
\pagestyle{fancy}

% LE: левая четная
% RO: правая нечетная
% CE, CO: центр четная, центр нечетная
\fancyhead[RO,LE]{\@lecture} % Right odd,  Left even
\fancyhead[RE,LO]{}          % Right even, Left odd

\fancyfoot[RO,LE]{\thepage}  % Right odd,  Left even
\fancyfoot[RE,LO]{}          % Right even, Left odd
\fancyfoot[C]{\leftmark}     % Center

\makeatother

\usepackage{todonotes} % Заметки
\usepackage{tcolorbox} % Цветные рамки

\tcbuselibrary{breakable} % Для многострочных рамок

% Поддержка рисунков
\usepackage{import}
\usepackage{xifthen}
\usepackage{pdfpages}
\usepackage{transparent}
\newcommand{\incfig}[1]{%
    \def\svgwidth{\columnwidth}
    \import{./figures/}{#1.pdf_tex}
}

% Fix some stuff
\pdfsuppresswarningpagegroup=1


% My name
\author{Belousov M.}